\documentclass[12pt]{report}

\renewcommand{\bibname}{References}

\usepackage{amsmath,amsfonts,amssymb,graphicx}
\oddsidemargin=0in
\evensidemargin=0in
\textwidth=6.25in
\headsep=0pt
\headheight=0pt
\topmargin=0in
\textheight=9.5in

\usepackage[utf8]{inputenc}
\usepackage[T1]{fontenc}
\usepackage[english]{babel}

\title{\vspace{-2.5cm}
\begin{center}
\includegraphics[width=2.5cm]{kth_cmyk_enginee_sciences.eps}\\[-1mm]
\hspace{-3mm} {\tiny {\sf Mechatronics}}
\end{center}
\vspace{5cm}
Los Cubleros}

\author{Mikael Sjöstedt (YYMMDD-XXXX) \\ Alexander Ramm (YYMMDD-XXXX)hejhejhej\\
{\tt Någon kontakt till oss. typ mail}\\
MF123X Degree Project in Mechatronics\\
Department of ITM vad det nu är \\
Royal Institute of Technology (KTH)\\
Supervisor: Daniel Frode \\
Examiner Martin Edin Grimheden}

\begin{document}

\maketitle

\begin{abstract}
This is the abstract. \\ You know nothing John Snow
\end{abstract}

\tableofcontents

\chapter{Introduction}

A general introduction to the context of the Bachelor thesis.

\chapter{Background Material}

A summary and presentation of the literature of the scientific field
({\it e.g.}  books, reports, theses, articles, and web sites) that you
have used and read in order to perform the thesis. For example, in
Ref.~\cite{einstein}, you can find the discovery of the law of the
photoelectric effect.

\chapter{Investigation}

Your investigation should consist of presenting the scientific problem
that you are studying, the theory or model that you are using, your
analytical calculations and approximations, your numerical analysis,
your results of the investigation, and finally, a discussion of your
results in a larger scientific context of the earlier literature on
your problem and/or field.

\section{Problem}

\section{Model}

\section{Analytical Calculations}

\section{Numerical Analysis}

\section{Results}

\section{Discussion}

\chapter{Summary and Conclusions}

This is the summary and conclusions.

\begin{thebibliography}{100}
\addcontentsline{toc}{chapter}{\bibname}
\bibitem{einstein} A.~Einstein, {\it Über einen die Erzeugung und
  Verwandlung des Lichtes betreffenden heuristischen Gesichtspunkt},
  Ann.~Phys.~\textbf{17}, 132-148 (1905).
\end{thebibliography}

\end{document}
